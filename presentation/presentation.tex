\documentclass{beamer}

\usetheme{metropolis}

\usepackage[utf8]{inputenc}
\usepackage[russian]{babel}
\usepackage[T2A]{fontenc}
\usepackage{graphicx}

\usebackgroundtemplate{
  \includegraphics[width=\paperwidth,height=\paperheight]{images/background.jpg}
}

\title{Кейс для продуктовых аналитиков (осень 2025)}
\author{Андрей}
\institute{Т-Образование}
\date{\today}

\begin{document}

\begin{frame}
  \titlepage
\end{frame}

\begin{frame}{Знакомство с датасетом}
    \begin{block}{О датасете}
        Данный датасет содержит подробную информацию о поездках на электросамокатах, арендованных через приложение Т-Банка. Каждый ряд таблицы описывает одну поездку и включает:
        \begin{itemize}
            \item параметры заказа (даты и время, стоимость, дальность, модель самоката);
            \item характеристики клиента (возраст, пол, образование, семейное положение);
            \item начисления и бонусы по лояльностным программам.
        \end{itemize}
        Датасет обезличен и предназначен для анализа пользовательского поведения, оптимизации тарифов и выявления закономерностей в использовании услуги аренды.
    \end{block}

    \vfill
    \small{Источник: \url{https://dano.hse.ru/data2024}}
\end{frame}

\begin{frame}{Распределение величин}
  \begin{figure}
    \includegraphics[width=\linewidth]{images/graph1.png}
  \end{figure}
\end{frame}

\begin{frame}{Основные выводы}
\begin{itemize}
    \item Объем данных: \textbf{396 749} наблюдений
    \item Средний возраст пользователей: \textbf{31.5 года}
    \begin{itemize}
        \item 75\% пользователей младше 37 лет
    \end{itemize}
    
    \item Типичная стоимость поездки:
    \begin{itemize}
        \item Активация: \textbf{42.6} (медиана: 50.0)
        \item Минута: \textbf{7.3} (медиана: 7.5)
        \item Итог: \textbf{85.9} (медиана: 78.0)
    \end{itemize}
    
    \item Стандартный залог: \textbf{300} (фиксированная величина для 75\% случаев)
    
    \item Средняя дистанция: \textbf{3.7 км} (медиана 1.7 км)
    \begin{itemize}
        \item Наличие экстремальных выбросов (макс: 56 012 км)
    \end{itemize}
    
    \item Время бронирования: \textbf{11.4 минуты} (медиана 7.8 минут)
    
    \item Лояльность пользователей:
    \begin{itemize}
        \item В среднем \textbf{36.5} поездок на пользователя
        \item Медианное значение \textbf{19} поездок
    \end{itemize}
\end{itemize}
\end{frame}

\begin{frame}{Проблемы данных}
\begin{alertblock}{Аномалии и выбросы}
\begin{itemize}
    \item Отрицательные значения в стоимости поездок
    \item Экстремальные расстояния (max: 56 012 км)
    \item Сильное отличие медианы от среднего по ключевым метрикам
\end{itemize}
\end{alertblock}

\begin{exampleblock}{Ключевые выводы}
\begin{itemize}
    \item Датчики расстояния электросамокатов моделей ES*** часто выходят из строя.
    \item Бонусы за импользование кредитной карты расчитываются не корректно или не очевидно.
    \item Надо бы все нормировать все при помощи \textbf{StandartScaler}, так мы не потеряем важные выбросы и сохраним основные отношения.
\end{itemize}
\end{exampleblock}
\end{frame}

\begin{frame}{Продуктовые гипотезы}
  \begin{itemize}
    \item Гипотеза 1: \textbf{Зависят ли предпочтения пользователей по тарифам от времени суток?}

Гипотеза предполагает, что краткосрочные поездки популярнее утром в рабочие дни.
    \item Гипотеза 2: \textbf{Влияет ли семейное положение человека на вероятность поездок?}

Проверка: Одинокие люди более склонны к поездкам.
    \item Гипотеза 3: \textbf{Как влияет расстояние поездки на выбор тарифа?}

Исследование: склонны ли пользователи выбирать дорогой тариф($>8.22$руб/мин) при коротких поездках ($<286$сек).
  \end{itemize}
\end{frame}

\begin{frame}{Гипотеза: Одинокие люди более склонны к поездкам}
\begin{columns}[T]
    
\begin{column}{0.48\textwidth}
\vspace{1em}
\textbf{Статистическая значимость}
\begin{itemize}
    \item t-тест: \textcolor{gray}{p=1.17e-18}
    \item ANOVA: \textcolor{gray}{p=2.9e-17}
    \item Mann-Whitney: \textcolor{gray}{p=9.97e-23}
    \item Kruskal-Wallis: \textcolor{gray}{p=2.32e-21}
\end{itemize}

\vspace{1em}
\textbf{Размер эффекта}
\begin{itemize}
    \item Cohen's d: 0.098 (\textcolor{gray}{малый})
    \item Eta²: 0.001 (\textcolor{gray}{0.1\% дисперсии})
\end{itemize}
\end{column}

\begin{column}{0.52\textwidth}
\vspace{1em}
\textbf{Среднее число поездок (95\% ДИ)}
\begin{itemize}
    \item \textbf{Одинокие:} 6.98 [6.75, 7.21]
    \item \textbf{Семейные:} 5.62 [5.42, 5.81]
    \item \textbf{Прочие:} 6.33 [5.62, 7.05]
\end{itemize}

\vspace{1em}
\textbf{Практическая значимость}
\begin{itemize}
    \item Разница: +24.2\% vs семейные
    \item Odds Ratio: 1.18
\end{itemize}
\end{column}

\end{columns}

\vspace{2em}
\begin{block}{Вывод}
\textbf{Гипотеза подтверждена:} одинокие люди действительно совершают поездки чаще семейных при статистической значимости во всех тестах, хотя эффект имеет небольшую величину.
\end{block}

\end{frame}

\begin{frame}{Рынок кикшеринга в России: консолидация и рост}
  \begin{columns}
    \begin{column}{0.48\textwidth}
      \begin{center}
        \includegraphics[width=1.1\linewidth]{images/market_share_pie_chart.png}
      \end{center}
      \begin{itemize}
        \item Топ-3 оператора контролируют \textbf{>85\%} рынка в денежном выражении
      \end{itemize}
    \end{column}
    \hfill
    \begin{column}{0.52\textwidth}
      \textbf{Динамично растущий рынок городской мобильности}
      \begin{itemize}
        \item Объём рынка в 2024 г.: \textbf{31,1 млрд руб.} (+39.8\% г/г)
        \item Количество поездок: \textbf{281,6 млн} 
        \item Аудитория: \textbf{25,3 млн} пользователей
      \end{itemize}
      
      \vspace{0.8cm}
      
    \end{column}
  \end{columns}
\end{frame}

\begin{frame}{Whoosh (Т-Банк) — абсолютный лидер рынка}
    
    \textbf{Рыночная доля:} \textbf{~47\%} от всех поездок в России \\
    \textit{(Каждый второй самокат в стране — Whoosh)}
    
    \vspace{0.3cm}
    
    \begin{columns}
        
        \column{0.5\textwidth}
        \textbf{Ключевые показатели (2024)}
        \begin{itemize}
            \item \textbf{149,7 млн} поездок \\ (+44\% г/г)
            \item Парк: \textbf{245,3 тыс.} самокатов
            \item География: \textbf{81} город \\ (Россия и Латинская Америка)
        \end{itemize}
        
        \column{0.5\textwidth}
        \textbf{Сравнение с конкурентами}
        \begin{itemize}
            \item \textit{Whoosh:} \textbf{~47\%}
            \item \textit{МТС Юрент:} \textbf{~33\%}
            \item \textit{Яндекс Go:} \textbf{~8\%}
            \item \textit{Прочие:} \textbf{~12\%}
        \end{itemize}
        
    \end{columns}
    
    \vspace{0.5cm}
    
    \centering
\end{frame}

\begin{frame}{ИИ-инструменты}
  Использовался Perplexity для
  \begin{itemize}
    \item Поиска информации
    \item Генерации громоздкого кода и верстки
    \item Определения доли рынка ТБанка
  \end{itemize}
  Так же благодарность Claude Sonet 4.5, DeepSeek, ChatGPT 5.
\end{frame}


\end{document}
